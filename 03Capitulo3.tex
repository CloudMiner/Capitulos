%---------------------------------------------------------------------
%
%                          Capítulo 3
%
%---------------------------------------------------------------------
\chapter{Tecnolog\'ias}

%\begin{FraseCelebre}
%\begin{Frase}
%...
%\end{Frase}
%\begin{Fuente}
%...
%\end{Fuente}
%\end{FraseCelebre}

%\begin{resumen}
%...
%\end{resumen}

%-------------------------------------------------------------------
%\section{Introducción}
%-------------------------------------------------------------------
\section{Tecnolog\'ias y librer\'ias utilizadas}
\subsection{Python}
\subsubsection{?`Qu\'e es Python?}
Python es un lenguaje de programaci\'on creado por Guido van Rossum a finales de 1980 cuyo nombre fue inspirado en el grupo de c\'omicos inlgeses \''Monty Python\''. Es un lenguaje de programaci\'on cuya filosof\'ia hace hincapi\'e en una sintaxis muy limpia y legible.\\

Se trata de un lenguaje interpretado, con tipado din\'amico, fuertemente tipado, orientado a objetos y multiplataforma

\subsubsection{?`Por qu\'e Python?}
Python es un lenguaje de programaci\'on que todo estudiante de ingenier\'ia deber\'ia conocer. Por las sus caracter\'istcas mencionadas anteriormente, es decir, por sintaxis simple, clara y sencilla; el tipado din\'amico, el gestor de memoria, la gran cantidad de librerías que dispone y la gran potencia del lenguaje hacen que desarrollar una aplicaci\'on en Python sea rápido, sencillo, con un c\'odigo limpio y legible.\\

Por todas estas caracter\'isticas, Python tiene un gran capacidad de integraci\'on en nuestro sistema CloudMiner respecto a otros lenguajes.

Algunos casos de \'exito en el uso de Python son Google, MSN, Yahoo, Wikipedia, Youtube, La NASA y todas las distribuciones de Linux.
\subsection{Web2py}
\subsubsection{?`Qu\'e es Web2py?}
Web2py es un framework de desarrollo web de c\'odigo abierto de muy f\'acil aprendizaje, incluye las últimas tecnolog\'ias de una forma simple y clara, tales como MVC, ORM, plantillas, java script, ajax, CSS, etc, lo que convierte en una soluci\'on completamente funcional para crear aplicaciones web de manera totalmente interactiva.

Su objetivo principal es dar soporte al desarrollo \'agil de software de apliaciones web escalables, seguras y portable, Web2py est\'a escrito en lenguaje Python.
\subsubsection{?`Por qu\'e Web2Py?}
Elegimos Web2py por una diversidad de factores, entre ellas podemos destacar.

Tiene una curva de aprendizaje muy llama respecto a otros frameworks y provee un entorno de desarrollo completamente basado en web.

Web2py tiene un foco en la seguridad prestando mecanismos predeterminados seguros, previniendo la las vulnerabilidades más comunes.



%-------------------------------------------------------------------
%\section*{\NotasBibliograficas}
%-------------------------------------------------------------------
%\TocNotasBibliograficas

%\medskip


%-------------------------------------------------------------------
%\section*{\ProximoCapitulo}
%-------------------------------------------------------------------
%\TocProximoCapitulo

% Variable local para emacs, para  que encuentre el fichero maestro de
% compilación y funcionen mejor algunas teclas rápidas de AucTeX
%%%
%%% Local Variables:
%%% mode: latex
%%% TeX-master: "../Tesis.tex"
%%% End:

